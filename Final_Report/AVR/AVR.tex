\documentclass[main.tex]{subfiles}
 
\begin{document}

\chapter[Spark V Programming]{Spark V Programming}
	\begin{enumerate}
		\item Setting up Spark V
		\item XBee interfacing with Spark V
		\item Go-to-Goal and PID controller
		\item Collision avoidance
	\end{enumerate}

Get the Spark V code on the git-hub repository \href{https://github.com/eYSIP-2017/eYSIP-2017_Formation_Control_of_Multiple_Swarm_Robots}{\ExternalLink}

	\section{Setting up Spark V}
	
	Refer the hardware \href{./datasheet/Spark V manuals/SPARK V ATMEGA16 Hardware Manual 2010-11-06.pdf}{\ExternalLink} and software \href{./datasheet/Spark V manuals/SPARK V ATMEGA16 Hardware Manual 2010-11-06.pdf}{\ExternalLink} manuals for using the Spark V robot.
		
		\subsection{Programming the Spark V using AVRDude}
			\begin{enumerate}
				\item Copy the AVRDude folder onto your computer
				\item Open the Command Prompt in the AVRDude folder
				\item Run - stkrun.bat ``SparkV.hex"   to program the Spark V with the file ``SparkV.hex".
					  Similarly you can program any file by giving the location of the file along with the file name.
			\end{enumerate}
		
		\subsection{Programming the Spark V using Atmel studio and AVRDude}
		\begin{enumerate}
			\item Copy the AVRDude folder onto your computer
			\item Open the \verb|Tools>External| tools in Atmel Studio
			\item Add STK500v2 as a tool
			\item Name - STK500v2
			\item Set the location of AVRDude.exe in the commands box\\
			 	  For eg \verb|C:\Users\Chirag\Desktop\eYSIP_2017\AVRDude\avrdude.exe|
			\item Arguments box - -c stk500v2 -p m16 -P NEX-USB-ISP -U flash:w:\verb|$(TargetName).hex:i|
			\item Initial Directory - \verb|$(TargetDir)|
			\item Go to \verb|tools>optinons>environment>keyboard|
			\item Select Tools.ExternalCommand1 and set any key as a shortcut (say alt+p) 
		\end{enumerate}
	You can now directly program Spark V from Atmel Studio using the shortcut.
						
	\pagebreak				
	\section{XBee interfacing with Spark V}
	
	The data is received character by character over UART. When a character is received it is very briefly stored in a buffer and a flag is set. This is done internally by the microcontroller. 
	This flag then triggers an interrupt.\\
	The interrupt then decides what to do with this character of data. If it is a valid byte of data it will be stored in the array \verb|"data_string_var"|. After a complete packet of data is received it is copied to the array \verb|"data_string"|.\\
	
	\verb|<#.........#>| is a proper string which then saved to \verb|"data_string"| by the ISR.\\
		
	\begin{lstlisting}[language=C, caption = Interrupt Service Routine]
		ISR(USART_RXC_vect)
		{		
			previous_data = data;
			data = UDR;
			
			strcpy((char*) data_string, (const char*)data_string_var); //Entire string received!! Save it!!
		
			if (previous_data==0x3C && data == 0x23)//< and #
			{
				append_on = 1;
				i=0;
			}
			
			else if (previous_data==0x23 && data==0x3E)//# and >
			{
				append_on=0;
				strcpy((char*)data_string, (const char*) data_string_var); //Entire string received!! Save it!!
			}
			
			else if (append_on==1 && data != 0x23)
			{
				data_string_var[i]=data;
				i++;
			}
		}
	\end{lstlisting}
	
	\section{Parsing the String}
	
	\begin{lstlisting}[language=C, caption = String Parsing Function]
	void update_values()
	{
		char parts[10][5];
		char data_string1[40];
		strcpy((char*)data_string1, (const char*)data_string);
		
		char *p_start, *p_end;
		unsigned char i=0;
		p_start = data_string1;
		
		while(1) 
		{
			p_end = strchr(p_start, '/');
			if (p_end)
			{
				strncpy(parts[i], p_start, p_end-p_start);
				parts[i][p_end-p_start] = 0;
				i++;
				p_start = p_end + 1;
			}
			else
			break;
		}
			x_current = atoi(parts[1]);
			y_current = atoi(parts[2]);
			theta_current = abs(atoi(parts[3])-360+180-360); //(0)-(360)
			x_req = atoi(parts[4]);
			y_req = atoi(parts[5]);
			theta_req = abs(atoi(parts[6])-180-360); //(0)-(360)
			trigger = atoi(parts[7]);
			trigger_angle = atoi(parts[8]);
	}
	\end{lstlisting}
	
	\noindent The string is received in the form
	\begin{verbatim}
	<#id/x_current/y_current/theta_current/x_required/y_required/theta_required/
	trigger/trigger_angle#>
	\end{verbatim}
	This is then parsed to obtain the values in int data type.
	
	\pagebreak	
	\section{Go-to-Goal and PID}
		\begin{lstlisting}[language=C, caption = PID controller]
			int v_left,v_right,R=3.5,L=11.5,V;
			float kp, ki, kd;
			float w;
			
			kp=.7;
			ki=0;
			kd=70;
			V=400;
		
			distance=sqrt(square(y_req-y_current)+square(x_req-x_current));
			destination=(distance<100)?1:0;
			
			if (!x_req || !y_req)
			{
				hard_stop();
				destination = 1;
				return;
			}
			
			V=(distance<100)?200:400;
			kp=(distance<100)?.4:.7;
			
			if(distance>20)
			{
				er=theta_current-theta_req; //-360 to 360
				er = atan2(sin(er*3.14/180), cos(er*3.14/180))*(180/3.14); //-180 to 180
				
				integral= integral+er*dt;
				derivative = (er-previous_er)/dt;
				previous_er = er;
				
				w=kp*er + ki*integral + kd*derivative;
				v_left=((2*V+w*L)/(2*R));
				v_right=((2*V-w*L)/(2*R));
				
				velocity(v_left,v_right);
				destination=(distance<100)?1:0;
			}		
			else
			{
				hard_stop();
			}
		\end{lstlisting}
		A Go-To-Goal PID controller is implemented on the microcontroller.\\
		The velocity of the motors is considered to be proportional to the PWM input to the wheels. The PID controller takes care of any non-linearity between the velocity and the PWM input.\\
		The velocity and the kp values are reduced as the robot nears its goal point to avoid overshoot and to avoid oscillations.\\
		Obstacle avoidance behavior is also turned off (by setting destination=1) as the robot nears its goal point so that it does not miss its goal point if another robot is near the goal point.
		
		\begin{verbatim}
		v_left=((2*V+w*L)/(2*R));
		v_right=((2*V-w*L)/(2*R));
		\end{verbatim}
		
		These equations convert the unicycle model to the differential drive model.\\
		The unicycle model gives the state of the robot in terms of its liner velocity 'V' (cm/sec) and its angular velocity 'w'.\\
		The differential drive model gives the state of the robot in terms of the angular velocity (RPM) of each wheels.\\
		While this equations are unnecessary in this case they could be made applicable if we had precise control over the angular velocity of the wheels(RPM).\\
		
		\pagebreak
		We use a modified function for velocity along with the Go-to-Goal controller. Negative velocity will make the wheel rotate in the opposite direction. Giving a value greater then 255 will have no effect and it will be capped to 255.
		
		\begin{lstlisting}[language=C, caption = velocity function]
		void velocity(int left_motor, int right_motor)
		{
			if (left_motor>=0 && right_motor>=0)
			{
				forward();
			}
			
			else if (left_motor<0 && right_motor>0)
			{
				left();
				left_motor=abs(left_motor)+40; //some offset is added so that the wheel still turns at lower values
			}
			
			else if (left_motor>0 && right_motor<0)
			{
				right();
				right_motor=abs(right_motor)+40;//some offset is added so that the wheel still turns at lower values
			}
			else 
			{
				back();
				left_motor=abs(left_motor);
				right_motor=abs(right_motor);
			}
			OCR1AL = left_motor;     // duty cycle 'ON' period of PWM out for Left motor
			OCR1BL = right_motor;     // duty cycle 'ON' period of PWM out for Right motor
		}
		\end{lstlisting}
		
	\pagebreak
	\section{Collision avoidance}
	
	\begin{lstlisting}[language = C, caption = Collision avoidance function]
		void avoid_obstacle()
		{
			if (trigger==1)
			{
				hard_stop();
				_delay_ms(100);
				velocity2(90, 90);
				if (trigger_angle > 50) back_mm(50);
				left_degrees(trigger_angle);
				forward_mm(50);
				hard_stop();
			}
			else if (trigger==2)
			{
				hard_stop();
				_delay_ms(100);
				velocity2(90, 90);
				if (trigger_angle > 50)	back_mm(50);
				right_degrees(trigger_angle);
				forward_mm(50);
				hard_stop();
			}
		}
	\end{lstlisting}
	
	If a trigger is received the robot executes this function. The robot reduces its speed, turns a specified number of degrees and moves forward. If the angle is greater then a certain value(50) the robot moves backwards before taking a turn. This is to avoid a head-on collision with another robot.  
			
\end{document}