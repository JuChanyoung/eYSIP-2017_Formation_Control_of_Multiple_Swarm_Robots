\section{Python and OpenCV setup}
	
	\section{Aruco Detection}
			Download and install the ArUco library from this git-hub repository \href{https://github.com/skvark/opencv-python}{OpenCV-contrib}\\
			Reference websites\\
			\href{http://docs.opencv.org/3.1.0/d5/dae/tutorial_aruco_detection.html}{OpenCV}\\
			\href{http://www.philipzucker.com/aruco-in-opencv/}{reference website}\\
			
			Refer the example codes for the usage of the ArUco library
		\subsection{Example Codes}
			\subsection{generate.py}
				Run the generate.py script to generate and save an aruco marker of a desired id and size.\\
			\subsection{aruco1.py}
				\begin{lstlisting}[language=Python, caption = Recognising the ArUco markers]
				
					# Capture frame-by-frame
					ret, frame = cap.read()
				
					#Convert frame to grayscale
					gray = cv2.cvtColor(frame, cv2.COLOR_BGR2GRAY)
					
					#5x5 aruco marker and upto 250 ids
					aruco_dict = aruco.Dictionary_get(aruco.DICT_5X5_250)
					parameters = aruco.DetectorParameters_create()
				
					# lists of ids and the corners beloning to each id
					corners, ids, _ = aruco.detectMarkers(gray, aruco_dict, parameters=parameters)
				\end{lstlisting}
				
				This script will detect the corners of a marker present in the frame.
				
				\pagebreak
				\begin{lstlisting}[language=Python, caption = Pose and Oriention Calculation]
					x_center = (corners[0][0][0][0] + corners[0][0][1][0] + corners[0][0][2][0] + corners[0][0][3][0])/4
					y_center = (corners[0][0][0][1] + corners[0][0][1][1] + corners[0][0][2][1] + corners[0][0][3][1])/4
					
					x1 = corners[0][0][0][0]
					x2 = corners[0][0][1][0]
					y1 = corners[0][0][0][1]
					y2 = corners[0][0][1][1]
					
					x2 = x2-x1
					y2 = -(y2-y1)
					
					angle = math.degrees(math.atan2(y2,x2)) #Result ranges from -180 to 180
					
					print angle, (x_center,y_center)		
				\end{lstlisting}
				
				The center of the marker is found out by simply averaging the x and the y coordinates of all the corners of an ArUco marker.\\
				The orientation of an ArUco marker is obtained by taking the relative position between two adjacent corners and taking tan inverse of the ratio of the x and y ordinates.
				
			\subsection{help.py}
				Run the help script in the example code folder for the list of all the functions available in the ArUco libray.\\
					
	\section{XBee Configuration}
		\subsection{API mode}
			\subsubsection{Coordinator}
			\subsubsection{End device}
		\subsection{AT mode}
			\subsubsection{Coordinator}
			\subsubsection{End device}
	\section{XBee Interfacing  with python}
			\subsection{Pyserial libraries}
			\subsection{XBee libraries}